\\documentclass{article}
\\usepackage[margin=1in]{geometry}
\\usepackage{graphicx}
\\usepackage{booktabs}
\\usepackage{lineno}
\\begin{document}
\\linenumbers

\\begin{center}
{\\LARGE Vegetation community composition and species-environment relationships along an elevational gradient in south-central Bhutan}\\par
\\end{center}

\\section*{Abstract}
\\textbf{Questions:} Understanding how plant communities vary along elevation is essential for predicting biodiversity responses to environmental change. We asked how community composition, diversity, and species--environment relationships differ among vegetation strata along an elevational gradient, and which environmental variables best explain regeneration patterns.\\par
\\textbf{Location:} [TODO -- STUDY LOCATION REQUIRED]\\par
\\textbf{Methods:} We analysed vegetation data from four strata (trees, shrubs, herbs, regeneration) across 209 to 221 plots. We constructed community matrices, quantified environmental variables, applied non-metric multidimensional scaling and canonical correspondence analysis, and tested group differences with permutational multivariate analysis of variance. Indicator species analysis and machine-learning models (random forest and gradient boosting) assessed regeneration patterns.\\par
\\textbf{Results:} Community composition differed among forest types in all strata (R2 0.0165 to 0.0502; p = 0.001). Environmental predictors explained 3.2 to 3.8 per cent of variation in trees and shrubs and were not significant for herbs and regeneration. Trees had the highest Shannon diversity (1.391 plus or minus 0.595), whereas herbs had the lowest (0.325 plus or minus 0.451). Random forest models outperformed gradient boosting for regeneration richness (root mean square error 1.165 plus or minus 0.182; coefficient of determination 0.142 plus or minus 0.040).\\par
\\textbf{Conclusions:} Plant communities show consistent elevational structuring across strata, with distinct environmental signals by layer. These results highlight the need to consider vertical heterogeneity when assessing mountain forest biodiversity and regeneration.\\par
\\textbf{Total word count:} 232\\par

\\noindent\\textbf{Keywords:} elevational gradient, plant community, tropical mountain forest, species--environment relationships, community assembly, diversity, regeneration, ordination, canonical correspondence analysis, indicator species

\\section*{Introduction}
Elevational gradients provide powerful natural laboratories for understanding how climate, topography, and habitat heterogeneity structure biodiversity. Across taxa and regions, elevational diversity patterns are often unimodal or vary with sampling extent and environmental context, indicating that no single mechanism universally explains richness--elevation relationships (McCain and Grytnes 2013). In mountain forests, strong abiotic gradients combine with fine-scale heterogeneity to shape community composition and to filter species differently across vegetation strata.\\par

In the eastern Himalaya, steep environmental gradients and complex terrain create sharp transitions among forest types and high beta diversity across short distances. Yet, community-level evidence that integrates multiple strata (trees, shrubs, herbs, regeneration) remains limited, and few studies explicitly compare the strength of species--environment relationships among strata or evaluate regeneration patterns alongside diversity and composition.\\par

Here, we examine plant community composition, diversity, and species--environment relationships along an elevational gradient in south-central Bhutan. We test whether forest types differ in composition across strata, identify environmental predictors of community structure, and quantify diversity patterns and indicator species. We also assess regeneration patterns using machine-learning models to identify key predictors. Together, these analyses provide a multi-strata perspective on elevational community structure and regeneration dynamics in subtropical to cool broadleaved forests.\\par

\\section*{Methods}
\\subsection*{Study area}
The study was conducted in unmanaged forests of Sarpang District, south-central Bhutan, within the jurisdiction of the Sarpang Forest Division. The region spans a strong altitudinal gradient from approximately 153 to 3 500 m a.s.l., encompassing subtropical broadleaved (100--500 m), warm broadleaved (500--1 500 m), and cool broadleaved forests (1 500--3 000 m). Annual precipitation ranges from 3 500 to 5 500 mm, driven by monsoonal circulation and orographic uplift, while steep terrain generates pronounced microclimatic heterogeneity. The landscape forms a strategic ecological link among Bhutan's protected areas and Biological Corridor-03, although most sampled forests remain outside formal management regimes.\\par

\\subsection*{Sampling design}
Vegetation sampling followed a stratified random design along the elevational gradient from Shershong (approximately 260 m) in the south to Singye (approximately 1 964 m) in the north. A total of 221 plots were established across forest types. Each plot measured 20 \\times 20 m (400 m\\textsuperscript{2}), with a nested 2 \\times 2 m subplot used to survey herbaceous vegetation. Fieldwork was undertaken between March and November to capture seasonal variation while avoiding peak monsoon inaccessibility.\\par

\\subsection*{Vegetation data collection}
All free-standing woody stems with diameter at breast height (DBH) > 1 cm and height > 1.3 m were recorded. DBH was measured using diameter tapes and tree height with hypsometers. Shrubs were assessed for presence, height, and lateral spread, whereas herbs were quantified using per cent cover, frequency, and maximum height. Species identifications were verified with regional floras and herbarium material. Species nomenclature followed the Flora of Bhutan and was cross-checked against the World Flora Online (accessed 6 Feb 2026).. Structural descriptors including basal area, DBH class, and height class distributions were derived to characterise stand structure and regeneration dynamics.\\par

\\subsection*{Environmental variables}
Topographic attributes (elevation, slope, aspect) were recorded in the field using GPS receivers, clinometers, and compasses. Spatial climatic surfaces for temperature, precipitation, evapotranspiration, and water balance were extracted from national-scale interpolation models and matched to plot coordinates. Digital elevation data from ASTER DEM were used to derive terrain predictors incorporated into statistical analyses.\\par

\\subsection*{Data preparation}
Field records were curated in tabular form to compute species abundance, frequency, basal area, and importance value index for each vegetation stratum. Alpha diversity was quantified using Shannon--Wiener diversity, Simpson’s index, and Pielou's evenness for each forest type and vegetation layer (Shannon 1948; Simpson 1949; Pielou 1966).\\par

\\subsection*{Multivariate community analyses}
Floristic variation among plots was examined using non-metric multidimensional scaling based on Bray--Curtis dissimilarities (Bray and Curtis 1957; Kruskal 1964). Species--environment relationships were assessed with canonical correspondence analysis using climatic and topographic predictors (ter Braak 1986). Ordination axes were interpreted through correlations between species scores and environmental variables. Community differences among forest types were tested using permutational multivariate analysis of variance (Anderson 2001).\\par

Vegetation communities were delineated by hierarchical agglomerative clustering, and solution robustness was evaluated using partitioning around medoids. Indicator species analysis identified taxa significantly associated with forest types at alpha = 0.05 using the indicator value approach (Dufrene and Legendre 1997). Sampling adequacy and dominance patterns were assessed through species--area curves and rank--abundance distributions.\\par

\\subsection*{Regeneration dynamics and predictive modelling}
Seedling and sapling data were analysed to quantify regeneration patterns and spatial structure. Kernel density estimation was applied to map recruitment hotspots across the landscape. Environmental drivers of regeneration density were modelled using random forest and XGBoost algorithms (Breiman 2001; Chen and Guestrin 2016), with predictor importance evaluated via gain, cover, and mean-decrease-accuracy metrics. Model performance was assessed using mean squared error and out-of-bag error rates. K-means clustering was used to classify plots into regeneration zones representing contrasting environmental regimes and recruitment intensity.\\par

\\subsection*{Software environment}
All analyses were conducted in R with community-ecology workflows implemented primarily through the vegan package (Oksanen et al. 2025). Multivariate classification and indicator-species procedures were cross-checked in PC-ORD v5, and spreadsheet software was used for initial data curation and quality control.\\par

\\subsection*{Reproducibility and data stewardship}
Reproducibility was ensured by archiving raw field data, spatial environmental layers, curated datasets, and derived analytical outputs; documenting all distance measures, clustering criteria, ordination settings, and machine-learning parameters; and recording software versions. Intermediate products including ordination scores, cluster assignments, regeneration surfaces, and variable-importance tables were retained to permit independent replication and re-analysis of vegetation patterns and regeneration models.\\par

\\section*{Results}
\\subsection*{Community composition and ordination}
Ordination results are shown in Figure 1. Environmental vectors significant at p < 0.05 are indicated.\\par

\\begin{figure}[h]
\\centering
\\includegraphics[width=0.9\\linewidth]{../02_figures/Figure01_NMDS.png}
\\caption{NMDS ordination of vegetation plots for four strata (trees, shrubs, herbs, regeneration). Bray--Curtis dissimilarity, k = 2. Red arrows indicate significant environmental vectors (envfit, p < 0.05, 999 permutations).}
\\end{figure}

\\subsection*{Community differences among forest types}
Community composition differed significantly among forest types across all strata (Table 2).\\par

\\begin{table}[h]
\\caption{PERMANOVA and dispersion tests by stratum.}
\\centering
\\begin{tabular}{l l r r r r r r r l}
\\toprule
Stratum & Group variable & Sites & Groups & R2 & F & p & Dispersion F & Dispersion p & Homogeneous dispersion \\
\\midrule
Trees & forest\_type & 216 & 3 & 0.0307 & 3.38 & 0.001 & 1.24 & 0.291 & TRUE \\
Shrubs & forest\_type & 194 & 3 & 0.0502 & 5.05 & 0.001 & 12.93 & 0.001 & FALSE \\
Herbs & forest\_type & 206 & 3 & 0.0165 & 1.70 & 0.001 & 15.05 & 0.001 & FALSE \\
Regeneration & forest\_type & 205 & 3 & 0.0208 & 2.14 & 0.001 & 1.17 & 0.319 & TRUE \\
\\bottomrule
\\end{tabular}
\\end{table}

\\subsection*{Species--environment relationships}
Canonical correspondence analysis indicated that selected environmental variables explained a small but significant proportion of variation for trees and shrubs, but not for herbs and regeneration (Table 3; Figure 4).\\par

\\begin{table}[h]
\\caption{Canonical correspondence analysis summary by stratum.}
\\centering
\\begin{tabular}{l r r l r r l r l}
\\toprule
Stratum & Sites & Species & Environmental variables & Total inertia & Constrained inertia & Proportion explained & p & Significant \\
\\midrule
Trees & 216 & 221 & aspect; eto; latitude; longitude; slope & 35.117 & 1.109 & 3.2 per cent & 0.008 & TRUE \\
Shrubs & 194 & 101 & aspect; eto; latitude; longitude; slope & 27.832 & 1.045 & 3.8 per cent & 0.001 & TRUE \\
Herbs & 206 & 134 & aspect; eto; latitude; longitude; slope & 68.475 & 2.309 & 3.4 per cent & 0.125 & FALSE \\
Regeneration & 205 & 109 & aspect; eto; latitude; longitude; slope & 58.906 & 1.987 & 3.4 per cent & 0.076 & FALSE \\
\\bottomrule
\\end{tabular}
\\end{table}

\\begin{figure}[h]
\\centering
\\includegraphics[width=0.9\\linewidth]{../02_figures/Figure04_CCA.png}
\\caption{CCA biplots for four vegetation strata. Environmental predictors selected after variance inflation factor screening (threshold = 10). Significance tested with 999 permutations.}
\\end{figure}

\\subsection*{Diversity patterns}
Alpha diversity differed among strata (Table 4; Figure 2), and species accumulation patterns are shown in Figure 3. Species richness varied along the elevational gradient (Figure 6).\\par

\\begin{table}[h]
\\caption{Alpha diversity by stratum.}
\\centering
\\begin{tabular}{l r l l r r}
\\toprule
Stratum & Plots & Richness (mean plus or minus SD) & Shannon & Simpson (mean) & Evenness (mean) \\
\\midrule
Herbs & 210 & 1.79 plus or minus 1.32 & 0.325 plus or minus 0.451 & 0.193 & 0.804 \\
Regeneration & 209 & 2.00 plus or minus 1.26 & 0.443 plus or minus 0.499 & 0.264 & 0.839 \\
Shrubs & 198 & 4.77 plus or minus 3.43 & 1.063 plus or minus 0.706 & 0.524 & 0.832 \\
Trees & 221 & 5.30 plus or minus 2.57 & 1.391 plus or minus 0.595 & 0.665 & 0.903 \\
\\bottomrule
\\end{tabular}
\\end{table}

\\begin{figure}[h]
\\centering
\\includegraphics[width=0.9\\linewidth]{../02_figures/Figure02_Diversity.png}
\\caption{Alpha diversity indices across vegetation strata. (A) Species richness, (B) Shannon--Wiener diversity, (C) Simpson’s index (one minus D), (D) Pielou's evenness.}
\\end{figure}

\\begin{figure}[h]
\\centering
\\includegraphics[width=0.9\\linewidth]{../02_figures/Figure03_SpeciesAccumulation.png}
\\caption{Species accumulation curves (random method, 100 permutations) showing cumulative species richness as a function of sampling effort for each vegetation stratum.}
\\end{figure}

\\begin{figure}[h]
\\centering
\\includegraphics[width=0.9\\linewidth]{../02_figures/Figure06_Diversity_Elevation.png}
\\caption{Species richness along the elevational gradient for each vegetation stratum. LOESS smoothing applied.}
\\end{figure}

\\subsection*{Indicator species}
Indicator species analysis identified significant indicators across all strata (Table 5).\\par

\\begin{table}[h]
\\caption{Indicator species summary by stratum.}
\\centering
\\begin{tabular}{l l r r r r l}
\\toprule
Stratum & Group variable & Sites & Species & Groups & Significant indicators & Per cent significant \\
\\midrule
Trees & elevation\_group & 216 & 221 & 3 & 10 & 4.5 per cent \\
Shrubs & elevation\_group & 194 & 101 & 3 & 15 & 14.9 per cent \\
Herbs & elevation\_group & 206 & 134 & 3 & 10 & 7.5 per cent \\
Regeneration & elevation\_group & 205 & 109 & 3 & 8 & 7.3 per cent \\
\\bottomrule
\\end{tabular}
\\end{table}

\\subsection*{Regeneration modelling}
Random forest models outperformed gradient boosting models for regeneration richness (Table 6; Figure 5). Top predictors included shrub Shannon diversity, shrub richness, temperature, elevation, and evapotranspiration.\\par

\\begin{table}[h]
\\caption{Cross-validated model performance for regeneration richness.}
\\centering
\\begin{tabular}{l r r l l l}
\\toprule
Model & Predictors & Observations & Coefficient of determination & Root mean square error & Mean absolute error \\
\\midrule
Random forest & 13 & 192 & 0.142 plus or minus 0.040 & 1.165 plus or minus 0.182 & 0.907 \\
Gradient boosting & 13 & 192 & -0.009 plus or minus 0.096 & 1.261 plus or minus 0.192 & 0.942 \\
\\bottomrule
\\end{tabular}
\\end{table}

\\begin{figure}[h]
\\centering
\\includegraphics[width=0.9\\linewidth]{../02_figures/Figure05_RF_Importance.png}
\\caption{Random forest variable importance for regeneration richness prediction. Top ten predictors ranked by percentage increase in mean squared error.}
\\end{figure}

\\section*{Discussion}
Community composition differed significantly among forest types across all strata, indicating strong structuring along the elevational gradient. The ordination and PERMANOVA results together suggest that shifts in forest type are associated with consistent compositional changes despite high within-type variability. Such stratified responses are consistent with the view that elevational gradients provide strong environmental filters but that local heterogeneity and dispersal processes contribute substantial additional variation (McCain and Grytnes 2013).\\par

Canonical correspondence analysis identified relatively small proportions of explained variation for trees and shrubs and non-significant models for herbs and regeneration. Low constrained inertia is common in community datasets where multiple unmeasured drivers and stochastic processes influence composition. The stronger signals in woody strata compared with herbs and regeneration suggest that longer-lived strata integrate environmental conditions over longer periods, whereas short-lived layers may respond to fine-scale microsite variability and recent disturbance.\\par

Diversity patterns varied by stratum, with trees showing the highest Shannon diversity and herbs the lowest. These differences likely reflect contrasting life-history strategies and the balance between dominance and evenness across strata. The indicator species results highlight that forest types retain distinctive floristic signatures, supporting their use in ecological classification and management.\\par

Regeneration modelling indicated that shrub-layer diversity and climate variables (temperature, elevation, and evapotranspiration) were among the strongest predictors of regeneration richness. This suggests that regeneration dynamics are influenced by both biotic context and environmental gradients, reinforcing the need to consider multi-strata interactions when assessing forest resilience and future compositional change.\\par

\\section*{Conclusions}
Plant communities along the Sarpang elevational gradient are strongly structured by forest type across all strata, with distinct species assemblages and indicator taxa. Environmental predictors explain modest but significant variation in woody strata, while herb and regeneration layers show weaker coupling to measured predictors. Regeneration richness is best predicted by a combination of shrub diversity and climatic variables, underscoring the importance of vertical heterogeneity for understanding forest dynamics. These findings provide a multi-strata baseline for monitoring and for assessing how Himalayan forest communities may respond to environmental change.\\par

\\\section*{Data availability statement}\nThe data supporting the findings of this study are available in the Supporting Information (Appendix S1--S10). Processed community matrices and analysis outputs are archived in the project repository accompanying this submission (see 03_analysis/06_stage3 and 03_analysis/04_results for traceable outputs). A public repository with DOI will be provided upon acceptance. [TODO — INSERT DOI WHEN AVAILABLE]\\par\n\n\\section*{References}
Anderson, Marti J. 2001. “A new method for non-parametric multivariate analysis of variance.” \\textit{Austral Ecology} 26(1): 32--46. https://doi.org/10.1111/j.1442-9993.2001.01070.pp.x.\\par

Bray, J. Roger, and J. T. Curtis. 1957. “An Ordination of the Upland Forest Communities of Southern Wisconsin.” \\textit{Ecological Monographs} 27(4): 325--349. https://doi.org/10.2307/1942268.\\par

Breiman, Leo. 2001. “Random Forests.” \\textit{Machine Learning} 45: 5--32. https://doi.org/10.1023/A:1010933404324.\\par

Chen, Tianqi, and Carlos Guestrin. 2016. “XGBoost: A Scalable Tree Boosting System.” In \\textit{Proceedings of the 22nd ACM SIGKDD International Conference on Knowledge Discovery and Data Mining}, 785--794. https://doi.org/10.1145/2939672.2939785.\\par

Dufrene, Marc, and Pierre Legendre. 1997. “Species assemblages and indicator species: The need for a flexible asymmetrical approach.” \\textit{Ecological Monographs} 67(3): 345--366. https://doi.org/10.2307/2963459.\\par

Kruskal, J. B. 1964. “Nonmetric multidimensional scaling: A numerical method.” \\textit{Psychometrika} 29(2): 115--129. https://doi.org/10.1007/BF02289694.\\par

McCain, Christy M., and John-Arvid Grytnes. 2013. “Global variation in elevational diversity patterns.” \\textit{Scientific Reports} 3: 3007. https://doi.org/10.1038/srep03007.\\par

Oksanen, Jari, Gavin L. Simpson, F. Guillaume Blanchet, Roeland Kindt, Pierre Legendre, Peter R. Minchin, R. B. O'Hara, Peter Solymos, M. Henry H. Stevens, Eduard Szoecs, Helene Wagner, Matt Barbour, Michael Bedward, Ben Bolker, Daniel Borcard, Tuomas Borman, Gustavo Carvalho, Michael Chirico, Miquel De Caceres, Sebastien Durand, Heloisa Beatriz Antoniazi Evangelista, Rich FitzJohn, Michael Friendly, Brendan Furneaux, Geoffrey Hannigan, Mark O. Hill, Leo Lahti, Cameron Martino, Dan McGlinn, Marie-Helene Ouellette, Eduardo Ribeiro Cunha, Tyler Smith, Adrian Stier, Cajo J. F. ter Braak, and James Weedon. 2025. \\textit{vegan: Community Ecology Package}. R package version 2.7-2. https://CRAN.R-project.org/package=vegan. https://doi.org/10.32614/CRAN.package.vegan.\\par

Pielou, E. C. 1966. “The measurement of diversity in different types of biological collections.” \\textit{Journal of Theoretical Biology} 13: 131--144. https://doi.org/10.1016/0022-5193(66)90013-0.\\par

Shannon, Claude E. 1948. “A Mathematical Theory of Communication.” \\textit{Bell System Technical Journal} 27: 379--423, 623--656. https://doi.org/10.1002/j.1538-7305.1948.tb00917.x.\\par

Simpson, E. H. 1949. “Measurement of Diversity.” \\textit{Nature} 163: 688. https://doi.org/10.1038/163688a0.\\par

ter Braak, C. J. F. 1986. “Canonical correspondence analysis: a new eigenvector technique for multivariate direct gradient analysis.” \\textit{Ecology} 67(5): 1167--1179. https://doi.org/10.2307/1938672.\\par

\\section*{Supporting information list}
Appendix S1. TableS1\_VIF\_all\_strata -- Variance inflation factor screening by stratum.\\par
Appendix S2. TableS2\_PERMANOVA\_full -- Full PERMANOVA and dispersion outputs.\\par
Appendix S3. TableS3\_envfit\_all -- Envfit results for all strata.\\par
Appendix S4. TableS4\_indicator\_species\_full -- Indicator species list.\\par
Appendix S5. TableS5\_ML\_parameters -- Machine-learning model parameters.\\par
Appendix S6. TableS6\_variable\_importance -- Variable importance outputs.\\par
Appendix S7. TableS7\_CV\_fold\_results -- Cross-validation fold results.\\par
Appendix S8. TableS8\_diversity\_correlations -- Diversity correlation matrix.\\par
Appendix S9. TableS9\_beta\_diversity -- Beta diversity metrics.\\par
Appendix S10. TableS10\_matrix\_diagnostics -- Community matrix diagnostics.\\par

\\end{document}





